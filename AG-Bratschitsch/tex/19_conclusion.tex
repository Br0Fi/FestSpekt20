\newpage
\section{Conclusion}
	% Rückgriff auf Hypothese und drittes Nennen dieser
	% Quellen zitieren, Websiten mit Zugriffsdatum
	% Verweise auf das Laborbuch (sind erlaubt)
	% Tabelle + Bilder mit Beschriftung

  % monolagen gefunden, Materialien konnten identifiziert werden (MoSe2 aber etwas unpassend)
  % SPE gefunden, Sättigung in Leistungsspektrum leider nicht

  In conclusion it can be said that it was possible to create monolayers of TMDCs using exfoliation and the materials used were identified within reasonable certainty by their respective exciton transition energy, though some deviation remained.

  It was also possible to find a single-photon emitter in a sample of hexagonal boron nitride by conducting antibunching measurements and the behavior of the emitter was quantified in dependence of incident laser power.
  The measurement of the efficiency of the emitter was incomplete, as the emitter didn't survive the measurement process.
  In the future it might be possible to reduce stress on the sample, by not performing several antibunching measurements of the same emitter at high laser powers, before measuring spectra.

  Lastly, the results regarding the faraday rotation are satisfying. Only one set of data points for the WS$_2$ monolayer at \SI{0.75}{\tesla} could have been retaken in hopes for a better fit.
  The influence of the excitons in the faraday rotation is strongly seen through the comparison of verdet constants.
