\section{Einleitung}
	% Hypothese	und deren Ergebnis, wenn Hypothese ist, dass nur Theorie erfüllt, sagen: Erwartung: Theorie aus einführung (mit reflink) erfüllt
	% Ergebnisse, auch Zahlen, mindestens wenn's halbwegs Sinn ergibt
	% Was wurde gemacht
	% manche leute wollen Passiv oder "man", manche nicht

  In diesem Versuch sollen verschiedene lose zusammenhängende Versuche an Festkörpern durchgeführt werden.

  Einerseits soll inverse Photoemission verwendet werden, um unbesetzte Zustände einer TiSe$_2$-Probe zu untersuchen.
  Konkret sollen hier sogenannte Ladungsdichtewellen untersucht werden.
  Hierfür wird zunächst eine Kalibrationsmessung anhand einer Kupfer-(111)-Oberfläche durchgeführt.
  Außerdem wird Low-Energy Electron Diffraction verwendet, um bei der schrittweisen Abtragung der Schutzbeschichtung durch Heizen anhand der Kristallstruktur festzustellen, wann die Schutzschicht vollständig entfernt wurde.

  Andererseits wird der magnetooptische Kerr-Effekt genutzt, um die Magnetisierung einer Probe in Abhängigkeit von einem äußeren Magnetfeld messen zu können.
  Hier wird festgestellt, dass die CoPt-Probe nur mit Magnetfeld senkrecht zur Probenebene magnetisierbar ist und die CrO$_2$-Probe nur mit Magnetfeldern in der Ebene.
  Es wird außerdem gezeigt, dass Remanenz und Koerzitivfeldstärke in der CoPt-Probe deutlich größer ist.
