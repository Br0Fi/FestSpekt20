\clearpage
\section{Inverse Photoemission}
\label{sec:prep}

%Warum TiS2? falls das hier war, i dont recall.

%Kupfer 111 OF für Winkelkalibrierung.
%Erst irgendwas sputtern (Argon, iirc). Aceton-Gas für Zählrohr.
%Kühlen für: der manipulator ist ziemlich maissv und da ist kupferblock drin. Wenn der zu warm wird, dampft der krass aus. (500°C für 1h).
%Schnitt senkrecht durch Bandschema. Zwei Messungen, eine nahe Null (Null ja erstmal nicht bekannt) da sollte eigentlich der niederenergetische Zustand verschwinden, weil unter Fermienergie, also nicht zugängig für IPE. Oberer Zustand ist Bildladungszustand.
%Da ist dann noch so zwei Zustände drüber. Die kann man nicht unterscheiden, aber Hoffnung wäre, die zu sehen. (Oder war das schon TiS2?)
%
%
%
