\newpage
\section{Schlussfolgerung}
	% Rückgriff auf Hypothese und drittes Nennen dieser
	% Quellen zitieren, Websiten mit Zugriffsdatum
	% Verweise auf das Laborbuch (sind erlaubt)
	% Tabelle + Bilder mit Beschriftung

  Zusammenfassend lässt sich sagen, dass mithilfe der inversen Photoemission zwar der markante Oberflächenzustand der Kupfer-(111)-Oberfläche sichtbar gemacht werden konnte.
  Die CDW-Zustände der TiSe$_2$-Probe konnten jedoch aufgrund mangelnder Kühlung der Probe nicht nachgewiesen werden.
  Low-Energy Electron Diffraction wurde verwendet, um den Unterschied zwischen amorphem Selen und dem TiSe$_2$-Kristall sichtbar zu machen.

  Der magnetooptische Kerr-Effekt konnte in beiden untersuchten Proben gezeigt werden und genutzt werden, um die Magnetisierung der Proben in Abhängigkeit von einem äußeren Magnetfeld zu messen.
  Es wurde festgestellt, dass die CoPt-Probe nur mit Magnetfeld senkrecht zur Probenebene magnetisierbar ist und die CrO$_2$-Probe nur mit Magnetfeldern in der Ebene und dass Remanenz und Koerzitivfeldstärke in der CoPt-Probe deutlich größer sind.

  Es würde in Zukunft Sinn ergeben, für die Messung der Magnetisierung weitere Dezimalstellen zu speichern, da Rundungsfehler einen signifikanten Einfluss auf das Ergebnis hatten.
