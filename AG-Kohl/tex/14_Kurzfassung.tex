\section{Einleitung}
	% Hypothese	und deren Ergebnis, wenn Hypothese ist, dass nur Theorie erfüllt, sagen: Erwartung: Theorie aus einführung (mit reflink) erfüllt
	% Ergebnisse, auch Zahlen, mindestens wenn's halbwegs Sinn ergibt
	% Was wurde gemacht
	% manche leute wollen Passiv oder "man", manche nicht

	Sowohl SEM als auch TEM sind wichtige bildgebende Verfahren, wenn die optische Auflösungsgrenze überschritten werden soll.
	Beide Verfahren sind jedoch über einen einfach Hell-Dunkel-Kontrast hinaus dazu in der Lage detaillierte Informationen über die elementare Zusammensetzung der Probe zu geben.

	In diesem Protokoll soll das SEM angewendet werden, um zunächst verschiedene mikroskopische Strukturen in makroskopischen Proben zu untersuchen.
	So werden unter anderem die Einzelaugen des Facettenauges einer Hummel betrachtet und Strukturen mit einer Größe von \SI{300}{nm} aufgelößt.
	Auf Kohlenstoff aufgedampfte Goldcluster mit einer Größe von etwa \SI{10}{nm} können nicht aufgelößt werden.

	Außerdem werden anhand von Referenzproben mittels EDX die Elementanteile von Kupfer und Zink in einer Messingprobe bestimmt und in einer Probe, auf der vier verschiedene Elemente aufgebracht wurden, eine Elementkarte erstellt.
	\par
	Das TEM wird angewendet, um eine Wolframdiselenidprobe zu untersuchen.
	So werden einerseits Hell- und Dunkelfeldaufnahmen angefertigt, die es erlauben, unterschiedliche Informationen über die Morphologie der Probe zu erhalten.
	Andererseits werden hochaufgelößte TEM-Bilder der Probe verwendet, um den Atomabstand in der Projektionsebene zu bestimmen.
	Diese Messung kann durch eine Aufnahme in der Streuebene bestätigt werden.

	Zuletzt wird ein EEL-Spektrum der Probe angefertigt.
	Hier kann der Plasmomen-Peak und ein zu Selen gehöriger Peak identifiziert werden.
	Andere Peaks können nicht erkannt werden.
