\clearpage
\section{Probenpräparation}

	Die Proben, welche für diese Untersuchung angefertigt werden, sollen aus Goldclustern (durchschnittlich ca. \SI{20}{\nano\meter} Durchmesser) auf einer dünnen Graphitschicht bestehen.
	Um solche Proben herzustellen, wird zunächst Kohlenstoff auf dünne Glasplatten aufgedampft.
	Anschließend werden die Glasplatten in destilliertes Wasser getaucht, sodass sich die dünnen Graphitschichten von ihnen lösen.
	Da diese Schichten nicht stabil genug sind, um in das Mikroskop eingebaut zu werden, werden dünne Kupfergitter mit 400 Maschen auf \SI{25,4}{\milli\meter\squared} verwendet. %TODO Angabe korrekt?
	Auf diese wird dann ein Stück der schwimmenden Graphitschicht gelegt.
	Nach dem Trocknen werden die Gitter mit dem Kohlenstoff in eine Halterung eingesetzt und der Aufdampfprozess wiederholt, nun jedoch mit Gold.
	Das Resultat sind die gewünschten Proben, da sich das Gold in Clustern auf dem Kohlenstoff ablagert.