\newpage
\section{SEM} % TODO

	Das für diese Untersuchung verwendete Rasterelektronenmikroskop detektiert mit einem SE-Detektor die Sekundärelektronen.
	Zusätzlich kann zeitgleich das Röntgenspektrum der untersuchten Probe aufgenommen werden.
	
	Zunächst werden verschiedene Beispielproben unter dem Mikroskop betrachtet und dann eine der zuvor hergestellten Proben untersucht, damit die Auflösungsgrenze bestimmt werden kann.
	Daraufhin wird das Spektrum einer Messingprobe aufgenommen und mit denen einer Kupfer- und einer Zinkprobe verglichen, um den jeweiligen Gehalt in dem Messing zu bestimmen.
	Zuletzt wird eine Probe mit vier unbekannten Elementen betrachtet.
	Dafür sollen die auftretenden Elemente anhand ihrer Röntgenspektren bestimmt werden und eine Karte der Elemente angefertigt werden.

\subsection{Beispielproben und Auflösungsgrenze} % TODO

	\
	
	...

\subsection{Untersuchung einer Messingprobe} % TODO

	Die zu untersuchende Messinprobe befindet sich wie auch die Kupfer- und die Zinkprobe an verschiedenen Stellen der gleichen Halterung.
	Beispielhaft ist dies in \cref{fig:messing_halterung} dargestellt.
	Um nur das Spektrum einer bestimmten Probe aufzunehmen, wird nur der Bereich, welcher von Interesse ist, abgebildet.
	Dafür wird die Probe innerhalb des Mikroskops in den Mittelpunkt des Elektronenstrahls gefahren und die Vergrößerung erhöht, damit nur die zu untersuchende Legierung bzw. das Element sichtbar ist.
	
	In \cref{fig:messing_spektrum} sind die aufgenommen Spektren für Messing, Kupfer und Zink zu erkennen.
	Ihre Charakteristischen Peaks entsprechen den %TODO Linien \cite{}
	Anhand der Amplituden von Legierung und reinen Elementen, lässt sich die Anteile letzterer in der Messingprobe schließen.
	Dafür werden Gauß-Anpassungen an den Peaks vorgenommen. % TODO erst Auswertung
	
	\
	
	...

\subsection{Untersuchung unbekannter Elemente} % TODO

	\
	
	...
		
	
	
