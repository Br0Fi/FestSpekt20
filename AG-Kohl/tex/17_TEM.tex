\newpage
\section{TEM} % TODO

 Es wird ein Libra 200 FE TEM (Zeiss) verwendet, um die in \cref{sec:prep} beschriebenen Proben zu untersuchen.
 Hierbei zeigt sich, dass die durchschnittlich etwa \SI{20}{nm} großen Goldcluster aufgelößt werden können.
 Aufgrund von Problemen mit dem Gerät werden jedoch im Folgenden ältere Messungen von Kohlenstoffnanoröhren ausgewertet.


 %TODO Bright-field vs. Dark-field, erklären an Fourierebene-Bild (Blende draufgelegt, etc.)

%Membran, Blick durch ein Loch.

%BF: Vakuum ist hell, je dicker desto dunkler.
%DF: Vakuum dunkel, Strukturen hell.

 %TODO HRTEM: Aus Bild Netzebenenabstand bestimmen und aus FT von Bild.

 %TODO Im low-loss die Peaks zuordnen (im coreloss nur Silizium vom Trägermaterial sichtbar)


 %die Multimap-Dateien sind EFTEM-Aufnahmen. Da sollen wir (?) nichts mit machen? TODO nachfragen.
