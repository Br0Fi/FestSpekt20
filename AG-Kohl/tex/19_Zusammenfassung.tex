\newpage
\section{Schlussfolgerung}
	% Rückgriff auf Hypothese und drittes Nennen dieser
	% Quellen zitieren, Websiten mit Zugriffsdatum
	% Verweise auf das Laborbuch (sind erlaubt)
	% Tabelle + Bilder mit Beschriftung
	%TODO schauen, ob hier was angepasst werden muss,

	Insgesamt lässt sich sagen, dass die verschiedenen Teilversuche erfolgreich durchgeführt werden konnten.
	Mit dem SEM konnte eine Auflösung von unter \SI{300}{nm} erreicht werden und anhand von EDX-Messungen Elementkonzentrationen in einer Messingprobe und Elementverteilungen in einer geteilten Probe bestimmt werden.
	Womöglich könnte im SEM eine bessere Auflösung durch Optimierung des Strahlengangs erreicht werden.
	Dies könnte beispielsweise durch Anpassung des Linsensystems und der Aberrationskorrekturen erreicht werden.
	Der damit verbundene Zeitaufwand hätte jedoch den Rahmen dieser Untersuchung überstiegen.

	Im TEM konnte eine Auflösung erreicht werden, die die Bestimmung der Abstände von Atomebenen ermöglicht.
	Diese wurden sowohl im Realraumbild als auch im Bild in der Streuebene bestimmt.

	Zuletzt wird das EEL-Spektrum der Probe aufgenommen und Plasmonen- und Selen-Peaks identifiziert.
